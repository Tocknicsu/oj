%!TEX TS-program = xelatex
%!TEX encoding = UTF-8 Unicode

\documentclass[12pt]{article}
\usepackage[bmargin=12in]{geometry}

\usepackage{fontspec}
\usepackage{color}
\usepackage[x11names]{xcolor}
\usepackage{xeCJK} 
\usepackage{amsmath, courier, listings, fancyhdr, graphicx}
%\topmargin=18pt
\headsep=5pt
\textheight=740pt
\footskip=0pt
\voffset=-42pt
\textwidth=545pt
\marginparsep=0pt
\marginparwidth=0pt
\marginparpush=0pt
\oddsidemargin=0pt
\evensidemargin=0pt
\hoffset=-42pt

\setmainfont{Consolas}              
\setmonofont{Consolas} 
\setCJKmainfont{文泉驛正黑} % Default CJK font
%\setCJKmainfont{NotoSansCJKtc-Regular}
\XeTeXlinebreaklocale "zh"                      
\XeTeXlinebreakskip = 0pt plus 1pt              




\lstset{                                            
    language=C++,                                   % the language of the code
    basicstyle=\footnotesize,                       % the size of the fonts that are used for the code
    numbers=left,                                   % where to put the line-numbers
    numberstyle=\footnotesize,                      % the size of the fonts that are used for the line-numbers
    stepnumber=1,                                   % the step between two line-numbers. If it's 1, each line  will be numbered
    numbersep=4pt,                                  % how far the line-numbers are from the code
    backgroundcolor=\color{white},                  % choose the background color. You must add \usepackage{color}
    showspaces=false,                               % show spaces adding particular underscores
    showstringspaces=false,                         % underline spaces within strings
    showtabs=false,                                 % show tabs within strings adding particular underscores
    frame=false,                                    % adds a frame around the code
    tabsize=2,                                      % sets default tabsize to 2 spaces
    captionpos=b,                                   % sets the caption-position to bottom
    breaklines=true,                                % sets automatic line breaking
    breakatwhitespace=false,                        % sets if automatic breaks should only happen at whitespace
    escapeinside={\%*}{*)},                         % if you want to add a comment within your code
        morekeywords={*},                               % if you want to add more keywords to the set
        keywordstyle=\bfseries\color{Blue1},
        commentstyle=\itshape\color{Red4},
        stringstyle=\itshape\color{Green4}
    }




\begin{document}

    \pagestyle{fancy}
    \fancyfoot{}
    %\fancyfoot[R]{\includegraphics[width=20pt]{ironwood.jpg}}
    \fancyhead[L]{國立交通大學 陳俊凱}
    \fancyhead[R]{\thepage}
    \renewcommand{\headrulewidth}{0.4pt}
    \renewcommand{\contentsname}{解題報告}

    \begin{center}\section{pA}\end{center}
    \input{./pa.txt}
    \lstinputlisting{./pa.cpp}
    \newpage

    \begin{center}\section{pB}\end{center}
    \input{./pb.txt}
    \lstinputlisting{./pb.cpp}
    \newpage

    \begin{center}\section{pC}\end{center}
    \input{./pc.txt}
    \lstinputlisting{./pc.cpp}
    \newpage

    \begin{center}\section{pD}\end{center}
    \input{./pd.txt}
    \lstinputlisting{./pdn.cpp}
    \newpage

    \begin{center}\section{pE}\end{center}
    \input{./pe.txt}
    \lstinputlisting{./pe.cpp}
    \newpage

    \begin{center}\section{pH}\end{center}
    \input{./ph.txt}
    \lstinputlisting{./ph.cpp}
    \newpage

    \begin{center}\section{pI}\end{center}
    \input{./pi.txt}
    \lstinputlisting{./pi.cpp}
    \newpage

    \begin{center}\section{pJ}\end{center}
    \input{./pj.txt}
    \lstinputlisting{./pj.cpp}
    \newpage
\end{document}
